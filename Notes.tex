\documentclass{article}
\usepackage[english]{babel}
\usepackage[utf8]{inputenc}
\usepackage{amsmath}
\usepackage{amsfonts}
\usepackage{listings}
\usepackage{xcolor}
\usepackage{parskip}
\usepackage{xurl}
\usepackage{hyperref}
\usepackage{graphicx}
\usepackage{enumitem}
\usepackage{float}
\usepackage[top = 1cm, bottom = 1cm, left = 1cm, right = 1cm]{geometry}


\graphicspath{ {./images/} }

\hypersetup{
    breaklinks=true,
    colorlinks=true,
    linkcolor=blue,
    filecolor=magenta,      
    urlcolor=blue,
    pdfpagemode=FullScreen,
}

\lstset{
basicstyle=\small\ttfamily,
columns=flexible,
breaklines=true
}



\urlstyle{same}

\title{Commodities Notes}
\author{Henry}


\begin{document}
\maketitle


\tableofcontents


\section{General/Derivatives}
\subsection{Futures}
\subsection{Swaps}
\subsection{Strategies}
\subsubsection{Calendar/Timespreads}
\subsubsection{Flys}
\subsubsection{Boxes}

\section{PnL calculation}
\subsection{Calculating PL change}
\subsection{Alternative ways to calculate PL change}
\subsubsection{Delta}
\subsection{Unit conversion}

Typically required when calculating MTM of a position when volumne unit of measure (UoM) is different from our marks/price unit of measure. Typically this happens in the cases of where we want to reflect product spread of 2 instruments with different volume UoM as a single positon. For example, we want to show a gasoil East/West (E/W) spread. Gasoil east (MOPS 10PPM Gasoil) default UoM is in bbls while Gasoil west (IPE Gasoil) default UoM is in mt. 

\subsubsection{General way of doing unit conversion}

unit conversion for volume is quite simple if  $$1mt = 7.45bbl$$ then to convert 50kb volume into its mts equivalent would be.

\begin{equation*}
    \begin{aligned}
        1bbl & = \frac{1}{7.45}mt \\ 
        50000bbl & = 50000 \times \frac{1}{7.45}mt \approx 6,711.4mt
    \end{aligned}    
\end{equation*}


In general 



\subsubsection{Unit conversion example}
\subsection{Combining positions}









\end{document}